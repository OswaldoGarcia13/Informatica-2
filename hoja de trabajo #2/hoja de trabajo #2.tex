\documentclass[10pt,a4paper]{article}
\usepackage[utf8]{inputenc}
\usepackage{amsmath}
\usepackage{amsfonts}
\usepackage{amssymb}
\begin{document}
\title{Hoja de Trabajo No.2}
\author{García Barrios, Oswaldo Jose}
\maketitle
\section*{Ejercicio \#1 (50\%)}
Demostrar utilizando inducci\'on:
\[
        \forall\ n.\ n^3\geq n^2
\]
\\donde $n\in\mathbb{N}$
\section*{Caso base:}
\ $0^3 > 0^2$
\\ P2 = todo numero natural unario tiene un sucesor diferente a él. 
\section*{Por inducción:}  
\ $S(n) = n + 1$
\\S (2) = 2 + 1 = 3 
\\S (2) = 3
\\ $ 3\mp 2  \Rightarrow$
\\ $N^s2 > n^2$
\\ $N^3 > N^2$
\section*{Ejercicio \#2 (50\%)}
Demostrar utilizando inducci\'on la desigualdad de Bernoulli:
\[
        \forall\ n.\ (1+x)^n\geq nx
\]
\\donde $n\in \mathbb{N}$, $x\in \mathbb{Q}$ y $x\geq -1$
\\
\\{\bf Consejo: }Es possible demostrar esto demostrando una propiedad m\'as fuerte
donde el lado izquierdo es mayor que $nx + 1$
\section*{Caso base:}
\ $(1 + x)^0 \geq 0*x$
\\ $ 1 \geq 0$
\section*{Por inducción: (n = - 1) }  
\ $(1+x)^{-1} \geq -1 * x$
\\ $\frac{1}{1+x} \geq -x$
\section*{Por inducción: (cualquier $n > 0$  ) }  
\ se realiza la demostración con el número 1, pues por el axioma de inducción dicha propiedad se pasará continuamente de sucesor en sucesor.
\\ $(1 + x)^1 \geq 1 * 1$
\\ $ (1 + x) \geq 1$




\end{document}