\documentclass[10pt,a4paper]{article}
\usepackage[utf8]{inputenc}
\usepackage{amsmath}
\usepackage{amsfonts}
\usepackage{amssymb}
\begin{document}
\title{Hoja de Trabajo No.1}
\author{García Barrios, Oswaldo Jose}
\maketitle
\section*{Ejercicio 2, Abstracción }
\subsection{conjunto de nodos}
\{1,2,3,4,5,6\}
\subsection{Conjunto de vertices }

    $$
        \left\langle \left\{
            \begin{bmatrix}
                \langle 1,2 \rangle & \langle 1, 3\rangle & \langle 1,4 \rangle \\
                \langle 1,5 \rangle & \langle 2,3 \rangle & \langle 2,4 \rangle \\
                \langle 2,6 \rangle & \langle 3,6 \rangle & \langle 3,5 \rangle \\
                \langle 4,6 \rangle & \langle 4,5 \rangle & \langle 5,6 \rangle \\
           
            \end{bmatrix}
        \right\}, 1, 6 \right\rangle
    $$ \\
\section*{Ejercicio 3}
\subsection{Respuestas}
\ >Qué estructura de datos podria representar un lanzamiento de dados?  

\ - Estructura de camino, pues cada lanzamiento siempre será iniciado en el número "1", por lo tanto dicho número actua como padre de las démas caras.

\ >Que algoritmo podriamos utilizar para generar dicha estructura?

\ - Sería un algoritmo que siempre tenga como arranque predeterminado iniciar en el número "1", a partir de esto se le pide que este generé las distintas caras del dado de manera aleatoria, al completar todas las caras, regresar al uno y reiniciar el algoritmo, asi no se crea un ciclo.

\ >Como nos aseguramos que ese algoritmo siempre produce un resultado?

\ - Estableciendo las caras del dado, de esa manera siempre tendrá que arrojar una de estas como resultado.
\end{document}